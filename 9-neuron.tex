\documentclass{article}
\usepackage{malmacros}
\begin{document}

\section{On Neurons, Biology and Cognition} % 1 person for whole 

\subsection{Qa The Biological Neurons and the Human Brain}
A neuron is a type of cell in the nerve system. It is different from other cells, since it is specializing in communication to other cells using synapses and information handling. See figure \ref{fig:les9-chemical-synapse-schema-cropped}.
\\ \\
Neurons are the primary components of the nervous system. It is made up of the central nervous system which for example includes the brain and spinal chord. They are responsible for receiving sensory input from the world and reacting to these accordingly by sending commands to the muscles. 
\\ \\
A neuron has a cell body (called a \textit{soma}), dendrites and an axon. Dendrites will expand from the soma, usually getting thinner and thinner. They are used to receive signals and send it down the axon. The axon is the connection to another neuron through neurons dendrite. They can also connect to another neurons axon. The signaling process is partly electrical and partly chemical. If the voltage changes by a large amount over a short interval, the neuron will generate an \textit{action potential} meaning a high pulse. This will active the synaptic connections when it reaches them. 
\\
The information flows between neurons using a \textit{synapse}. This is a small gap between the neurons. A synapse contains receptor sites for neurotransmitters to be able to transmit data between neurons.

\dsdfig{les9-chemical-synapse-schema-cropped}{8cm}{A signal propagating down an axon to the cell body and dendrites of the next cell}{H}

%Then explain, again with the wording of an engineer, not a biologist, how the neurons are structured in the brain. 

The structure of the neurons in the brain can be seen in figure \ref{fig:les9-gyrus-dentatus-40x}. The figure is a snapshot of the human hoppocampal tissue, where the neurons are Golgi-stained. The lumps are called \textit{Nissi bodies}. They are a large granular body found in neurons, and are involved in protein synthesis. Their prominence is due to their activitiy, as they are very metabolically active. It is obvious to see that these form the structure of the brain and are connected.

\dsdfig{les9-gyrus-dentatus-40x}{10cm}{Golgi-stained neurons in human hippocampal tissue}{H}

\noindent
References
\\
https://en.wikipedia.org/wiki/Neuron/media/File:Chemical\_synapse\_schema\_cropped.jpg
\\
https://en.wikipedia.org/wiki/Neuron/media/File:Gyrus\_Dentatus\_40x.jpg
\\
https://en.wikipedia.org/wiki/Cognition
\\
https://faculty.washington.edu/chudler/synapse.html
\\
https://da.wikipedia.org/wiki/Neuron
\subsection{Qb On Cognition}

This section explores the question "Will a machine ever be able to have cognition" given the following definition:

\begin{displayquote}
Cognition is "the mental action or process of acquiring knowledge and understanding through thought, experience, and the senses..."
\end{displayquote}

\noindent
Cognition can simply refer to the act of thinking. This is all the forms of thinking, from doing simple logical tasks to understand and recognize more complex emotions such as understanding or empathizing  with others. Cognition is not a static thing, it will change with time and with new experiences. Humans are very different, and have been shaped through experiences.
\\ \\
Even for humans, some decisions are hard to take, even when the choice would be obvious for a computer because of ethical reasons. A lot of our thinking can be seen as simple problems to solve, since most beings will see something as correct or wrong. Though, this is not applicable to more complex thoughts. It could be as  brutal as: should the car run over the child on the street, or drive into a family on the side walk? This choice is extremely hard for anyone to take. A computer may see this as: "what will the damage  be? How can I reduce the damage?". So is it even okay if a computer makes this choice? And if it does, who is even to blame for the consequences?
\\ \\
It is hard to say if a computer can eventually react as a human brain if it has enough neurons. The brain is not a simple machine, and there are many factors to consider, not just the logical thinking that a computer performs very well. When we as humans think, we will consider many things. We will consider other people, look at the best outcome, but not always follow 'the best decision' ourselves, because of some other underlying problem. Not one human is alike, so how can a computer be made to replicate a human brain? It may be able to replicate the 'ideal human' brain, but it will not be able to think like an actual human brain. This requires a personality and experience. Our way of thinking is based on our life, on our upbringing, experiences, beliefs and so on. To be able to replicate this, a machine needs to be able to take all of this into consideration.

%Now, explaining cognition is still a pretty impossible philosophical question to answer. So you should answer this question: what is your understanding of cognition?

%If we reproduce biological neurons, and just make enough of them, in your opinion, will a machine ever be able to have cognition?






\end{document}